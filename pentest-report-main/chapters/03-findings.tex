\chapter{Befunde}

%#####################################################################
\cvss{av=adjacent, ac=low, pr=low, ui=none, s=changed, c=high, i=high, a=high}
\cvssdescription{Etiam risus sapien, ornare at dui ut, semper eleifend arcu. In fermentum felis ut ornare convallis. Donec ultrices condimentum neque ut semper. Aenean sit amet purus a sapien sodales mollis. Donec felis mauris, eleifend et laoreet ut, consectetur a nulla. Morbi pretium accumsan convallis. Nullam pulvinar id nisl non tempus. In eget ullamcorper urna, in ultrices urna. Aenean nec pretium magna.}

\section{\makecvssbadge Lorem ipsum}
\cvssaddtosummary{Lorem ipsum}

%\makecvssbox

\paragraph{Zusammenfassung} Lorem ipsum dolor sit amet, consectetur adipiscing elit. Donec facilisis imperdiet augue ut molestie. Vivamus ultricies mi at lorem vulputate auctor. Etiam quis mi interdum, pretium dolor nec, hendrerit purus. Donec purus orci, pharetra non hendrerit sit amet, aliquam non tortor. Quisque hendrerit est quis risus molestie dictum. Donec scelerisque mauris lectus, non vulputate risus consequat ut. Sed vel faucibus dui.

\paragraph{Auswirkungen}
Praesent ornare porta eros, eu volutpat neque pulvinar volutpat. Suspendisse lobortis erat at odio maximus, sit amet mattis nisi condimentum. Ut sodales quam vitae sapien finibus commodo. Donec a neque vel dolor mollis tristique pulvinar id nibh. In vitae augue dignissim velit aliquam porta quis quis urna. Fusce lacus urna, accumsan non nulla a, efficitur convallis mauris. Suspendisse vitae mauris pulvinar, luctus leo ac, condimentum neque. Nunc malesuada ante lacus, a tempus est auctor eu. Sed sagittis ut urna nec gravida. Aenean nisi felis, accumsan convallis tortor venenatis, accumsan rutrum velit. Phasellus vel est tellus. Quisque consectetur vel metus vitae iaculis. Nam neque nibh, venenatis at neque ut, dapibus lobortis nisi. 

\paragraph{Proof of concept}
Interdum et malesuada fames ac ante ipsum primis in faucibus. Ut sagittis orci quis ipsum tristique tincidunt. Maecenas dignissim dapibus dolor, in aliquet erat feugiat sed. Suspendisse luctus diam feugiat, porttitor massa non, tempus elit. Nam tristique, mauris a facilisis scelerisque, magna arcu ornare ipsum, in rhoncus lectus ipsum dictum risus. In sit amet libero dolor. Etiam elementum ligula vel metus placerat, eget laoreet eros elementum. Maecenas eleifend neque odio, ac ultrices sapien vehicula quis. Proin mauris erat, pharetra lobortis odio non, facilisis finibus risus. Quisque ac enim ut libero dictum suscipit. Cras quam lectus, porta eget nisl interdum, faucibus commodo elit. Duis in tellus dapibus est sagittis aliquet. Maecenas justo elit, vestibulum eget urna nec, elementum fermentum elit. Nullam non dolor odio.

\paragraph{Empfehlungen}
Nullam cursus mattis efficitur. Etiam mauris urna, tristique eu sodales sed, luctus eget urna. Nam a semper risus, at porttitor ex. Nam gravida augue et pulvinar iaculis. Cras hendrerit sem at tortor volutpat luctus. Vestibulum ut aliquam elit. Nullam finibus, odio vel ornare maximus, diam leo blandit est, eu commodo nunc tellus vel diam. Phasellus ac hendrerit neque, ut ullamcorper lorem. Nunc at lobortis dui, et blandit augue. Aliquam accumsan nibh eu diam ullamcorper, sed euismod lorem mollis. Curabitur dictum, est vitae ultricies suscipit, nisi orci ornare lectus, a varius magna nibh sit amet est. Duis molestie erat et tellus sollicitudin sollicitudin. Nam dolor enim, facilisis vestibulum tellus ac, blandit varius justo. Nulla consequat mauris lectus, eget condimentum massa lobortis quis. 

\pagebreak


%#####################################################################
\cvss{av=network, ac=high, pr=low, ui=required, s=unchanged, c=high, i=low, a=none}
\cvssdescription{Die Apache-Version auf dem Server ist veraltet und weist Sicherheitslücken auf. Ein Angreifer könnte diese Lücken ausnutzen, um unautorisierten Zugriff zu erlangen oder den Server zu kompromittieren. Es wird empfohlen, die Apache-Version auf die neueste stabile Version zu aktualisieren.}

\section{\makecvssbadge Veraltete Apache Version}
\cvssaddtosummary{Veraltete Apache Version}


\paragraph{Zusammenfassung} Die auf dem Server installierte Apache-Version ist veraltet und gefährdet die Sicherheit des Systems.

\paragraph{Auswirkungen}
Die Verwendung einer veralteten Apache-Version erhöht das Risiko von Sicherheitsverletzungen, da bekannte Sicherheitslücken nicht behoben sind.

\paragraph{Proof of concept}
Die Apache-Version kann mithilfe von Tools wie \textit{nmap} oder durch Überprüfung der Serverkonfiguration identifiziert werden.

\paragraph{Empfehlungen}
Aktualisieren Sie die Apache-Version auf die neueste stabile Version, um bekannte Sicherheitslücken zu schließen und die Sicherheit des Servers zu gewährleisten. Führen Sie regelmäßig Sicherheitsupdates durch, um zukünftige Sicherheitsrisiken zu minimieren.
rsus mattis efficitur. Etiam mauris urna, tristique eu sodales sed, luctus eget urna. Nam a semper risus, at porttitor ex. Nam gravida augue et pulvinar iaculis. Cras hendrerit sem at tortor volutpat luctus. Vestibulum ut aliquam elit. Nullam finibus, odio vel ornare maximus, diam leo blandit est, eu commodo nunc tellus vel diam. Phasellus ac hendrerit neque, ut ullamcorper lorem. Nunc at lobortis dui, et blandit augue. Aliquam accumsan nibh eu diam ullamcorper, sed euismod lorem mollis. Curabitur dictum, est vitae ultricies suscipit, nisi orci ornare lectus, a varius magna nibh sit amet est. Duis molestie erat et tellus sollicitudin sollicitudin. Nam dolor enim, facilisis vestibulum tellus ac, blandit varius justo. Nulla consequat mauris lectus, eget condimentum massa lobortis quis. 


\pagebreak


%#####################################################################
\cvss{av=network, ac=high, pr=low, ui=required, s=unchanged, c=high, i=low, a=none}
\cvssdescription{Die Apache-Version auf dem Server ist veraltet und weist Sicherheitslücken auf. Ein Angreifer könnte diese Lücken ausnutzen, um unautorisierten Zugriff zu erlangen oder den Server zu kompromittieren. Es wird empfohlen, die Apache-Version auf die neueste stabile Version zu aktualisieren.}

\section{\makecvssbadge Veraltete Apache Version}
\cvssaddtosummary{Veraltete Apache Version}


\paragraph{Zusammenfassung} Die auf dem Server installierte Apache-Version ist veraltet und gefährdet die Sicherheit des Systems.

\paragraph{Auswirkungen}
Die Verwendung einer veralteten Apache-Version erhöht das Risiko von Sicherheitsverletzungen, da bekannte Sicherheitslücken nicht behoben sind.

\paragraph{Proof of concept}
Die Apache-Version kann mithilfe von Tools wie \textit{nmap} oder durch Überprüfung der Serverkonfiguration identifiziert werden.

\paragraph{Empfehlungen}
Aktualisieren Sie die Apache-Version auf die neueste stabile Version, um bekannte Sicherheitslücken zu schließen und die Sicherheit des Servers zu gewährleisten. Führen Sie regelmäßig Sicherheitsupdates durch, um zukünftige Sicherheitsrisiken zu minimieren.
